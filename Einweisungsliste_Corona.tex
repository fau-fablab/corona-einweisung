%%%%%%%%%%%%%%%%%%%%%%%%%%%%%%%%%%%%%%%%%%%%%%%%
% COPYRIGHT: (C) 2012-2015 FAU FabLab and others
% Bearbeitungen ab 2015-02-20 fallen unter CC-BY-SA 3.0
% Sobald alle Mitautoren zugestimmt haben, steht die komplette Datei unter CC-BY-SA 3.0. Bis dahin ist der Lizenzstatus aller alten Bestandteile ungeklärt.
%%%%%%%%%%%%%%%%%%%%%%%%%%%%%%%%%%%%%%%%%%%%%%%%


\newcommand{\basedir}{fablab-document}
\documentclass[13pt]{\basedir/fablab-document}
\usepackage{fancybox} %ovale Boxen für Knöpfe
\usepackage{amssymb} % Symbole für Knöpfe
\usepackage{subfigure,caption}
\usepackage{eurosym}
\usepackage{tabularx} % Tabellen mit bestimmtem Breitenverhältnis der Spalten
\usepackage{wrapfig} % Textumlauf um Bilder
\renewcommand{\texteuro}{\euro}
\usepackage{ifthen}
\usepackage{xspace}
\def\tabularnewcol{&\xspace} % hässlicher Workaround von http://tex.stackexchange.com/questions/7590/how-to-programmatically-make-tabular-rows-using-whiledo


\usepackage{tabularx} % Tabelle mit teilweise gleich großen Spalten
\title{Einweisungsliste Corona-Regeln}
\fancyfoot[C]{github.com/fau-fablab/corona-einweisung}
\fancyfoot[L]{Einweisungsliste Nr. Corona-20-\underline{\hspace{3em}}}

\begin{document}
%\maketitle

Ich bestätige mit meiner Unterschrift verbindlich, dass ich die Unterweisung zur \emph{Betriebsanweisung zum Schutz vor einer Infektion durch Coronavirus SARS-CoV-2} im FAU FabLab

\begin{itemize}
\item gelesen und verstanden habe
\item und beachten werde.
\end{itemize}


\newline
\newline


% einfach kopiert von Einweisungsliste Lasercutter

\newcounter{i}
\setcounter{i}{1}

\newcommand{\leerezeile}{\hspace{2em} \tabularnewcol \hspace{3em} \tabularnewcol \hspace{2.5em} \tabularnewcol \hspace{2.5em} \tabularnewcol \vbox{\vspace{2em}} \tabularnewcol \tabularnewline \hline}

\begin{tabularx}{\textwidth}{|l|l|l|l|X|X|}
  \hline
  \textbf{Nr.} & \textbf{Datum} & \textbf{von} & \textbf{bis} & \textbf{Name} & \textbf{Unterschrift} \\ \hline
  \whiledo{\value{i}<14}%
  {%
    \stepcounter{i} \leerezeile
  }%
  \leerezeile % doofer Workaround, eigentlich sollte das auch in der Forschleife gehen! Ohne dies wird die Spaltenbegrenzung von Spalte 1 zu weit gezeichnet.
\end{tabularx}

\end{document}

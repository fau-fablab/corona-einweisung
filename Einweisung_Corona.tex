%%%%%%%%%%%%%%%%%%%%%%%%%%%%%%%%%%%%%%%%%%%%%%%%
% COPYRIGHT: (C) 2012-2020 FAU FabLab and others
% Bearbeitungen ab 2015-02-20 fallen unter CC-BY-SA 3.0
%%%%%%%%%%%%%%%%%%%%%%%%%%%%%%%%%%%%%%%%%%%%%%%%


\newcommand{\basedir}{fablab-document}
\documentclass[13pt]{\basedir/fablab-document}


\date{2022}
%\pagestyle{empty} % keine Seitennummern
%\sffamily
\author{kontakt@fablab.fau.de}
\title{Corona-Unterweisung}
\linespread{1} % Zeilenabstand

\usepackage{parskip} % Abstände zwischen Absätzen / Listenelementen
 \setlength{\parskip}{0.7\parskip}
% \setlength{\parsep}{0pt}
% \setlength{\headsep}{0pt}
% \setlength{\topskip}{0pt}
% \setlength{\topmargin}{0pt}
% \setlength{\topsep}{0pt}
% \setlength{\partopsep}{0pt}

% \usepackage[compact]{titlesec} % Abstände bei Überschriften
% \titlespacing{\section}{0pt}{*0.5}{*0}
% \titlespacing{\subsection}{0pt}{*0.3}{*0}
%\titlespacing{\subsubsection}{0pt}{*0}{*0}

%\fancyfoot[L]{}
%\fancyfoot[C]{}
%\fancyfoot[R]{Version 2, April 2012}

\setcounter{secnumdepth}{0}

% solche Wörter werden nicht getrennt!
\hyphenation{Mechanikwerkstatt}
\hyphenation{Elektrowerkstatt}

% Neuer Befehl \subscript (Text tiefgestellt) von http://anthony.liekens.net/index.php/LaTeX/SubscriptAndSuperscriptInTextMode
%\newcommand{\subscript}[1]{\ensuremath{_{\textrm{\small{#1}}}}}

\begin{document}

\maketitle

\begin{center}
  \textbf{Unterweisung zum Infektionsschutz im FAU FabLab}
\end{center}

\vbox{\vspace{1cm}}

\section{Generelle Verhaltensregeln für Alle}
Es gelten stets die aktuellen von der bayerischen Staatsregierung und der Universität festgelegten Regelungen. \\
Diese werden wie folgt spezifiziert: 

\begin{itemize}
  \item  Starke Empfehlung zum Tragen einer möglichst gut schützenden Maske
  \item  Abstand halten wo immer möglich
  \item  Möglichkeiten zum Lüften nutzen (Dachlüftung, Dachfenster, Fenster)
  \item  Allgemein bekannte Hygieneregeln, wie Händewaschen, Hust-Nies-Etikette, etc. umsetzen
  \item  Wir nutzen weiterhin Terminreservierung per RSVP zur Kapazitätssteuerung um Überfüllung der Räume und Überlastung der Betreuer zu vermeiden.
\end{itemize}


\ccLicense{corona-unterweisung}{Corona-Unterweisung}

\end{document}

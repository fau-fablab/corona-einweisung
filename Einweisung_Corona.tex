%%%%%%%%%%%%%%%%%%%%%%%%%%%%%%%%%%%%%%%%%%%%%%%%
% COPYRIGHT: (C) 2012-2020 FAU FabLab and others
% Bearbeitungen ab 2015-02-20 fallen unter CC-BY-SA 3.0
%%%%%%%%%%%%%%%%%%%%%%%%%%%%%%%%%%%%%%%%%%%%%%%%


\newcommand{\basedir}{fablab-document}
\documentclass[13pt]{\basedir/fablab-document}


\date{2020}
%\pagestyle{empty} % keine Seitennummern
%\sffamily
%\author{Philipp, Max, Patrick} %ausgetauscht durch:
\author{kontakt@fablab.fau.de}
\title{Corona-Unterweisung}
\linespread{1} % Zeilenabstand

\usepackage{parskip} % Abstände zwischen Absätzen / Listenelementen
 \setlength{\parskip}{0.7\parskip}
% \setlength{\parsep}{0pt}
% \setlength{\headsep}{0pt}
% \setlength{\topskip}{0pt}
% \setlength{\topmargin}{0pt}
% \setlength{\topsep}{0pt}
% \setlength{\partopsep}{0pt}

% \usepackage[compact]{titlesec} % Abstände bei Überschriften
% \titlespacing{\section}{0pt}{*0.5}{*0}
% \titlespacing{\subsection}{0pt}{*0.3}{*0}
%\titlespacing{\subsubsection}{0pt}{*0}{*0}

%\fancyfoot[L]{}
%\fancyfoot[C]{}
%\fancyfoot[R]{Version 2, April 2012}

\setcounter{secnumdepth}{0}

% solche Wörter werden nicht getrennt!
\hyphenation{Mechanikwerkstatt}
\hyphenation{Elektrowerkstatt}

% Neuer Befehl \subscript (Text tiefgestellt) von http://anthony.liekens.net/index.php/LaTeX/SubscriptAndSuperscriptInTextMode
%\newcommand{\subscript}[1]{\ensuremath{_{\textrm{\small{#1}}}}}

\begin{document}

\maketitle

\begin{center}
  \textbf{Unterweisung zur \emph{Betriebsanweisung zum Schutz vor einer Infektion durch Coronavirus SARS-CoV-2} im FAU FabLab und zum Verhalten im Präsenzbetrieb}
\end{center}

\vbox{\vspace{1cm}}


\section{Generelle Verhaltensregeln für Alle}
Es gelten stets die in der aktuellen 
\emph{Betriebsanweisung zum Schutz vor einer Infektion durch Coronavirus SARS-CoV-2} des Sachgebiets Arbeitssicherheit der FAU, 
der \emph{Bayerischen Infektionsschutzmaßnahmen-Verordnung BayIfSMV}, 
dem \emph{Hygienekonzept der FAU} und 
die in diesen Dokumenten referenzierten Anweisungen 
festgelegten Regelungen. Diese werden wie folgt ergänzt beziehungsweise spezifiziert: 

\begin{itemize}
  \item  Dauerhafte Pflicht zum Tragen einer Maske im Kontakt mit Anderen (mehr als eine Person im Fablab). Ausser es wird mit Chemikalien oder offen drehenden Maschinen hantiert, dann entfällt die Pflicht.
  \item  Abstand halten wo immer möglich
  \item  Möglichkeiten zum Lüften nutzen (Dachlüftung, Dachfenster, Fenster)
  \item  Allgemein bekannte Hygieneregeln, wie Händewaschen, Hust-Nies-Etiktte, etc. umsetzen
  \item  Kontaktdatenerfassung für alle Anwesenden per darfichrein.de \footnote{Kontaktdatenerfassung standardmässig per darfichrein.de, ausser Papierformular gewünscht, dann FAU-Formular \emph{Analoge Kontaktdatenerfassung}, das in verschlossenem und beschriftetem Brief an das Sekretariat (Geschäftsstelle EEI) gegeben wird. Bei Besuchern reicht auch der Check-In im RSVP} und 3G-Regel. 
  \item  Obergrenze von 7 plus eventuell 3 Personen in den Räumen (Werkstattbereich plus eventuell Besprechungsraum nebenan) beachten
\end{itemize}


\section{Präsenzbetrieb in Öffnungszeiten}
Derzeit können wir unten den für Universitäten geltenden Regelungen wieder öffnen, was konkret derzeit bedeutet, dass Zugang nach 3G möglich ist (Personen, die vollständig geimpft, genesen oder aktuell negativ getestet sind).
Neben den generellen Verhaltensregeln orientieren wir uns in der Umsetzung an den bzw. gelten für uns die geltenden Regelungen für kleine Präsenzlehrveranstaltungen bzw Praktika.


Die Regelungen u.a. der \emph{Checkliste „3G-Regel“ an der FAU} setzen wir konkret (neben den oben angegebenen generellen Verhaltensregeln) wie folgt um:
\begin{itemize}
  \item  Betreuer des Öffnungszeiten sind für Kontrolle der 3G-Regel laut Checkliste der FAU und der Regeln wie Einhaltung von Abstand und Tragen der Masken verantwortlich.
  \item  Unterweisung aller Anwesenden in die geltenden Regeln
  \item  Für alle Anwesenden gilt die 3G-Regel (Ausnahmen, z.B. Kinder unter 6 Jahren, siehe \emph{Hygienekonzept der FAU} bzw. \emph{BayIfSMV} )
  \item  Maximalzahl der anwesenden Personen (siehe oben) gilt inklusive Betreuer
  \item  Besuchersteuerung durch RSVP-Plugin im Fablab-Kalender (Besucher sollen vorab Termine buchen, wodurch Anzahl begrenzt wird und die Besucher mit gebuchtem Termin Vorrang vor Spontanbesuchern haben.)
  \item  Kontaktdatenerfassung für alle Anwesenden (siehe oben) (darfichrein.de, Papierformular oder Check-In im Terminbuchungssystem RSVP)
\end{itemize}



\section{Modus der Unterweisung}
Alle Nutzer:innen des FAU Fablab mit Schliessberechtigung erhalten diese Unterweisung (Betriebsanweisung mit diesem Ergänzungsdokument) per Email und unterschreiben bei der ersten Nutzung des FAU Fablab die in den Räumen befindliche Liste und bestätigen damit das Gelesen-und-Verstanden-Haben und dass die Maßnahmen umgesetzt werden.
Alle Betreuer:innen werden zusätzlich auf ihre Pflichten (3G-Kontrolle und vertrauliche Behandlung der Daten der Besucher im Rahmen der Kontaktdatenerfassung und 3G-Kontrolle.)
Alle Besucher werden bei Betreten der Räumlichkeiten von anwesenden Betreuern in diese Regelungen eingewiesen.

\ccLicense{corona-unterweisung}{Corona-Unterweisung}

\end{document}

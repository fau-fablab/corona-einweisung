%%%%%%%%%%%%%%%%%%%%%%%%%%%%%%%%%%%%%%%%%%%%%%%%
% COPYRIGHT: (C) 2012-2020 FAU FabLab and others
% Bearbeitungen ab 2015-02-20 fallen unter CC-BY-SA 3.0
%%%%%%%%%%%%%%%%%%%%%%%%%%%%%%%%%%%%%%%%%%%%%%%%


\newcommand{\basedir}{fablab-document}
\documentclass[13pt]{\basedir/fablab-document}


\date{2020}
%\pagestyle{empty} % keine Seitennummern
%\sffamily
%\author{Philipp, Max, Patrick} %ausgetauscht durch:
\author{kontakt@fablab.fau.de}
\title{Corona-Unterweisung}
\linespread{1} % Zeilenabstand

\usepackage{parskip} % Abstände zwischen Absätzen / Listenelementen
 \setlength{\parskip}{0.7\parskip}
% \setlength{\parsep}{0pt}
% \setlength{\headsep}{0pt}
% \setlength{\topskip}{0pt}
% \setlength{\topmargin}{0pt}
% \setlength{\topsep}{0pt}
% \setlength{\partopsep}{0pt}

% \usepackage[compact]{titlesec} % Abstände bei Überschriften
% \titlespacing{\section}{0pt}{*0.5}{*0}
% \titlespacing{\subsection}{0pt}{*0.3}{*0}
%\titlespacing{\subsubsection}{0pt}{*0}{*0}

%\fancyfoot[L]{}
%\fancyfoot[C]{}
%\fancyfoot[R]{Version 2, April 2012}

\setcounter{secnumdepth}{0}

% solche Wörter werden nicht getrennt!
\hyphenation{Mechanikwerkstatt}
\hyphenation{Mechanikwerkstatt}

% Neuer Befehl \subscript (Text tiefgestellt) von http://anthony.liekens.net/index.php/LaTeX/SubscriptAndSuperscriptInTextMode
%\newcommand{\subscript}[1]{\ensuremath{_{\textrm{\small{#1}}}}}

\begin{document}

\maketitle

\begin{center}
  \textbf{Unterweisung zur \emph{Betriebsanweisung zum Schutz vor einer Infektion durch Coronavirus SARS-CoV-2} im FAU FabLab}
\end{center}

\vbox{\vspace{1cm}}


\section{Verhaltensregeln}
Es gelten stets die in der aktuelle \emph{Betriebsanweisung zum Schutz vor einer Infektion durch Coronavirus SARS-CoV-2} des Sachgebiets Arbeitssicherheit der FAU (hängt aus), der \emph{Bayerischen Infektionsschutzmaßnahmen-Verordnung} et cetera festgelegten Regelungen. Diese werden wie folgt ergänzt beziehungsweise spezifiziert: 

\begin{itemize}
  \item  Dauerhafte Pflicht zum Tragen einer Mund-Nasen-Bedeckung im Kontakt mit Anderen (mehr als eine Person im Fablab) ausser es wird mit Chemikalien oder offen drehenden Maschinen hantiert. Dann aber Abstand zu Anderen.
  \item  Möglichkeiten zum Lüften nutzen (Dachlüftung, Dachfenster, Fenster)
  \item  Bis von der Universität wieder Publikumsverkehr erlaubt wird und dann von uns ein Konzept dazu erstellt wurde: Besucher-Nutzung nur für dringende Projekte für Forschung und Lehre. Anmeldung und Terminvereinbarung per Mail auf der Mailingliste fablab-aktive.
  \item  Zur Kontaktminimierung nach Möglichkeit Nutzungen vorher im Kalender auf der Website als \emph{Lab belegt} mit Namen eintragen, so dass Andere wissen, dass das FabLab bereits belegt ist.
\end{itemize}

\section{Modus der Unterweisung}
Alle Nutzer:innen des FAU Fablab mit Schliessberechtigung erhalten diese Unterweisung (Betriebsanweisung mit diesem Ergänzungsdokument) per Email und bestätigen bei der ersten Nutzung des FAU Fablab die in den Räumen befindliche Liste und bestätigen das Gelesen- und Verstanden-Haben.

\ccLicense{corona-unterweisung}{Corona-Unterweisung}

\end{document}
